\documentclass[a4paper,12pt]{article}
\usepackage{amsmath}

\begin{document}


\noindent Problem 2:   What are the two's complement representations for the following (decimal) numbers? Show your work. Submit a tex file or equivalent (word, pages, etc) on your github directory "Homework 1". 
\newline
\newline
a) 10\\
10 represented in binary system is 1010. since it is positive number, the two's complement of it is still 1010.
\newline
\newline
b) 436\\
436 in binary system is 110110100 ($256+128+32+16+4=2^8+2^7+2^5+2^4+2^2$), and again it's also the value of its two's complement.
\newline 
\newline 
c) 1024\\
1024 in binary system is 10000000000 ($1024=2^{10}$), and it's also the value of its two's complement.
\newline  
\newline
d) -13\\
13 in binary system is 1101 ($8+4+1$). flip 1101 to get 0010. The next step is adding 1 to the flipped result 0010, $0010+1=0011$. Since the two's complement of negative number always start with 1, we need to add 1 to the left end. So the  two's complement of -13 is 10011.
\newline 
\newline
e) -1023\\
1024 in binary system is 10000000000. Since 1023=1024-1, so 1023 in binary system is 1111111111. Repeat the precedure in d): flip, add 1 , check and make the left end is 1. Then we can obtain the two's complement of -1023 which is 10000000001.
\newline  
\newline
f) -1024\\
1024 in binary system is 10000000000. Repeat the same precedure we can obtain the two's complement of -1024 which is 10000000000. (since 11bits can storage from -1024 to 1023, -1024 has two's complement representation in 11bits.) 
\end{document}
